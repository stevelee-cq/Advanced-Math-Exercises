%导言区
\documentclass[10pt]{article}
\usepackage{ctex}
\usepackage{geometry}
\usepackage{tcolorbox}
\geometry{b5paper,scale=0.8}
\usepackage{zhlipsum}
\usepackage{amsmath}
\title{极限计算}
\author{言午}
\date{\today}
% 本文档命令
\usepackage{array}
\newcommand{\ccr}[1]{\makecell{{\color{#1}\rule{1cm}{1cm}}}}
% 修改目录深度
\setcounter{tocdepth}{2}

\begin{document}
	\maketitle
	
	\textbf{1.}计算
	$$
	\lim_{n\rightarrow \infty} \frac{1}{n}\sum_{k=1}^n{\sqrt{1+\frac{k}{n}}}
	$$
	
	\textbf{2.}计算
	$$
	\lim_{n\rightarrow \infty} \frac{1^p+2^p+\cdots +n^p}{n^{p+1}}
	$$
	
	\textbf{3.}计算
	$$
	\lim_{n\rightarrow \infty} \frac{1}{n}\sum_{k=1}^n{\frac{1}{3^k}\left( 1+\frac{1}{k} \right) ^{k^2}}
	$$
	
	\textbf{4.}计算
	$$
	\lim_{x\rightarrow 0} \frac{1-\cos x\sqrt{\cos 2x}\cdots \sqrt[n]{\cos nx}}{x^2}
	$$
	
	\textbf{5.}计算
	$$
	\lim_{x\rightarrow 0} \frac{\tan\tan x-\sin\sin x}{\tan x-\sin x}
	$$
	
	\textbf{6.}计算
	\begin{equation*}
		\lim_{x\rightarrow 0} \frac{\ln \left( e^{\sin x}+\sqrt[3]{1-\cos x} \right) -\sin x}{\text{arc}\tan\left( 4\sqrt[3]{1-\cos x} \right)}
	\end{equation*}
	
	\textbf{7.}计算
	$$
	\lim_{x\rightarrow +\infty} \frac{\int_1^x{\left[ t^2\left( e^{\frac{1}{t}}-1 \right) -t \right] dt}}{x^2\ln \left( 1+\frac{1}{x} \right)}
	$$
	
	\textbf{8.}计算
	$$
	\lim_{x\rightarrow 0} \frac{\cos x-\cos 3x}{x^2}
	$$
	
	\textbf{9.}计算
	$$
	\lim_{n\rightarrow \infty} \sin \left( n\sqrt{4n^2+1}\pi \right)
	$$
	
	\textbf{10.}计算
	$$
	\lim_{n\rightarrow \infty} n\sin \sqrt{4n^2+1}\pi
	$$
	
	\textbf{11.}计算
	$$
	\lim_{x\rightarrow 0} \frac{\sqrt{1+x}-\sqrt{1-x}}{\sqrt[3]{1+x}-\sqrt[3]{1-x}}
	$$
	
	\textbf{12.}设$f\left( x \right)$在$x=0$的某邻域内可导,且在$x=0$处存在二阶导数,又设$f\left( 0 \right) =f'\left( 0 \right) =0$,$f''\left( 0 \right) \ne 0$,求极限
	$$
	I=\lim_{x\rightarrow 0} \frac{\int_0^x{\left( x-t \right) f\left( t \right) dt}}{x\int_0^x{f\left( x-t \right) dt}}
	$$
	
	\textbf{13.}已知$f\left( x \right)$在$x=0$处连续,计算
	$$
	\lim_{x\rightarrow 0} \frac{\int_0^x{\left( x-t \right) f\left( t \right) dt}}{x\int_0^x{f\left( x-t \right) dt}}
	$$
	
	\textbf{14.}计算
	$$
	\lim_{x\rightarrow 1} \frac{\left( 1-\sqrt[3]{x} \right) \left( 1-\sqrt[4]{x} \right) \cdots \left( 1-\sqrt[n]{x} \right)}{\left( 1-x \right) ^{n-2}}
	$$
	
	\textbf{15.}计算
	$$
	\lim_{n\rightarrow \infty} \left( n! \right) ^{\frac{1}{n^2}}
	$$
	
	\textbf{16.}计算
	$$
	\lim_{n\rightarrow \infty} \ln \sqrt[n]{\left( 1+\frac{1}{n} \right) ^k\left( 1+\frac{2}{n} \right) ^k\cdots \left( 1+\frac{n}{n} \right) ^k}
	$$
	
	\textbf{17.}计算
	$$
	\lim_{n\rightarrow \infty} \frac{1}{n^4}\prod_{i=1}^{2n}{\left( n^2+i^2 \right) ^{\frac{1}{n}}}
	$$
	
	\textbf{18.}计算
	$$
	\lim_{n\rightarrow \infty} \frac{1!+2!+\cdots +n!}{n!}
	$$
	
	\textbf{19.}设函数$f\left( x \right)$在$\left( -1,1 \right)$内有二阶连续导数,且$f''\left( x \right) \ne 0$,证明
	
	(1)对于$\left( -1,1 \right)$内任意$x\ne 0$,存在唯一的$\theta \left( x \right) \in \left( 0,1 \right)$,使
	$$
	f\left( x \right) =f\left( 0 \right) +xf'\left( \theta \left( x \right) x \right)
	$$
	
	(2)
	$$
	\lim_{x\rightarrow 0} \theta \left( x \right) =\frac{1}{2}
	$$
	
	\textbf{20.}计算
	$$
	\lim_{n\rightarrow \infty} n\left( \frac{1}{n^2+\pi}+\frac{1}{n^2+2\pi}+\cdots +\frac{1}{n^2+n\pi} \right)
	$$
	
	\textbf{21.}计算
	$$
	\lim_{x\rightarrow 0^+} x\left[ \frac{1}{x} \right]
	$$
	
	\textbf{22.}计算
	$$
	\lim_{x\rightarrow 0} \frac{\sin x-\tan x}{\left( \sqrt[3]{1+x^2}-1 \right) \left( \sqrt{1+\sin x}-1 \right)}
	$$
	
	\textbf{23.}$n$为正整数,$a$为某实数$a\ne0$,且
	$$
	\lim_{x\rightarrow +\infty} \frac{x^{1999}}{x^n-\left( x-1 \right) ^n}=\frac{1}{a}
	$$
	
	则$n$、$a$的值分别为多少。
	
\end{document}