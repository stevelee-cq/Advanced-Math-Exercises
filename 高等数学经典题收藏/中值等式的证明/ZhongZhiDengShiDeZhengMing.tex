%导言区
\documentclass[10pt]{article}
\usepackage{ctex}
\usepackage{geometry}
\usepackage{tcolorbox}
\geometry{b5paper,scale=0.8}
\usepackage{zhlipsum}
\usepackage{amsmath}
\title{中值等式的证明}
\author{言午}
\date{\today}
% 本文档命令
\usepackage{array}
\newcommand{\ccr}[1]{\makecell{{\color{#1}\rule{1cm}{1cm}}}}
% 修改目录深度
\setcounter{tocdepth}{2}

\begin{document}
\maketitle

\textbf{1.}设函数$f\left( x \right)$在$\left[ a,b \right]$上连续,在$\left( a,b \right)$内可导,证明至少存在一点$\xi \in \left( a,b \right)$,使得
$$
\frac{f\left( \xi \right) -f\left( a \right)}{b-\xi}=f'\left( \xi \right)
$$

\textbf{2.}设函数$f\left( x \right)$在$\left[ a,b \right]$上连续,在$\left( a,b \right)$内可导,且$f\left( a \right) =f\left( b \right) =1$,证明存在$\xi ,\eta \in \left( a,b \right)$,使得
$$
e^{\eta -\xi}\left[ f\left( \eta \right) +f'\left( \eta \right) \right] =1
$$

\textbf{3.}设函数$f\left( x \right)$在$\left[ a,b \right]$上连续,在$\left( a,b \right)$内可导,$f'\left( x \right) \ne 0$,且$f\left( a \right) =0$,$f\left( b \right) =2$,证明在区间$\left( a,b \right)$内存在两个不同的点$\xi$、$\eta$使得
$$
f'\left( \eta \right) \left[ f\left( \xi \right) +\xi f'\left( \xi \right) \right] =f'\left( \xi \right) \left[ bf'\left( \eta \right) -1 \right]
$$

\textbf{4.}设函数$f\left( x \right)$在$\left[ a,b \right]$上连续,在$\left( a,b \right)$内二阶可导,证明存在$\xi \in \left( a,b \right)$,使得
$$
f\left( b \right) -2f\left( \frac{a+b}{2} \right) +f\left( a \right) =\frac{\left( b-a \right) ^2}{4}f''\left( \xi \right)
$$

\textbf{5.}设$f\left( x \right)$在$\left[ a,b \right]$上连续,$f\left( a \right) =f\left( b \right)$,证明在区间$\left[ a,b \right]$上存在$\xi$,使得
$$
f\left( \xi \right) =f\left( \xi +\frac{b-a}{2} \right)
$$

\textbf{6.}设函数$f\left( x \right)$在$\left[ 0,1 \right]$上三阶可导,且$f\left( 0 \right) =-1$,$f\left( 1 \right) =0$,$f'\left( 0 \right) =0$,证明$\forall x\in \left( 0,1 \right)$,至少存在一点$\xi \in \left( 0,1 \right)$,使得
$$
f\left( x \right) =-1+x^2+\frac{x^2\left( x-1 \right)}{3!}f'''\left( \xi \right)
$$

\textbf{7.}设函数$f\left( x \right)$在$\left[ 0,1 \right]$上三阶可导,且$f\left( 0 \right) =f\left( 1 \right) =0$,又$F\left( x \right) =x^3f\left( x \right)$,证明存在$\xi \in \left( 0,1 \right)$,使$F'''\left( \xi \right) =0$.

\textbf{8.}设函数$f\left( x \right)$在$\left[ 0,3 \right]$上连续,在$\left( 0,3 \right)$内可导,且$f\left( 0 \right) +f\left( 1 \right) +f\left( 2 \right) =3$,$f\left( 3 \right) =1$,证明至少存在一点$\xi \in \left( 0,3 \right)$,使得$f'\left( \xi \right) =0$.

\textbf{9.}函数$f\left( x \right)$在$\left[ 0,1 \right]$上连续,$f\left( 0 \right) =f\left( 1 \right)$,证明对于任意自然数$n$,存在$\xi \in \left[ 0,1 \right)$,使得
$$
f\left( \xi +\frac{1}{n} \right) =f\left( \xi \right)
$$

\textbf{10.}设函数$f\left( x \right)$在$\left[ a,b \right]$上可导,$\mu$为介于$f'\left( a \right)$与$f'\left( b \right)$之间的实数,则存在$\xi \in \left( a,b \right)$,使
$$
f'\left( \xi \right) =\mu
$$

\textbf{11.}函数$f\left( x \right)$在$\left[ a,b \right]$上二阶连续可导,证明存在$\xi \in \left( a,b \right)$,使得
$$
\int_a^b{f\left( x \right) dx}=\frac{1}{2}\left[ f\left( a \right) +f\left( b \right) \right] \left( b-a \right) -\frac{1}{12}f''\left( \xi \right) \left( b-a \right) ^3
$$

\textbf{12.}函数$f\left( x \right)$在$\left[ a,b \right]$上二阶连续可导,且$f'\left( a \right) =f'\left( b \right)$,证明存在$\xi \in \left( a,b \right)$,使得
$$
\int_a^b{f\left( x \right) dx}=\frac{1}{2}\left[ f\left( a \right) +f\left( b \right) \right] \left( b-a \right) +\frac{1}{6}f''\left( \xi \right) \left( b-a \right) ^3
$$

\textbf{13.}函数$f\left( x \right)$在$\left[ a,b \right]$上二阶连续可导,且$f'\left( a \right) =f'\left( b \right)$,证明存在$\xi \in \left( a,b \right)$,使得
$$
\int_a^b{f\left( x \right) dx}=\frac{1}{2}\left[ f\left( a \right) +f\left( b \right) \right] \left( b-a \right) +\frac{1}{24}f''\left( \xi \right) \left( b-a \right) ^3
$$

\end{document}