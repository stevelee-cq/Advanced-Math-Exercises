%导言区
\documentclass[10pt]{article}
\usepackage{ctex}
\usepackage{geometry}
\usepackage{tcolorbox}
\geometry{b5paper,scale=0.8}
\usepackage{zhlipsum}
\usepackage{amsmath}
\title{递推式极限}
\author{言午}
\date{\today}
% 本文档命令
\usepackage{array}
\newcommand{\ccr}[1]{\makecell{{\color{#1}\rule{1cm}{1cm}}}}
% 修改目录深度
\setcounter{tocdepth}{2}

\begin{document}
	\maketitle
	
	\textbf{1.}设$a_1=1$,$a_{n+1}=a_n+\frac{1}{a_n}\left( n\in N_+ \right)$,证明
	$$
	\lim_{n\rightarrow \infty} \frac{a_n}{\sqrt{n}}=\sqrt{2}
	$$
	
	\textbf{2.}设数列$\left\{ x_n \right\}$满足$x_1=2$,$x_{n+1}=2+\frac{1}{x_n}$,$n\in N_+$,证明数列$\left\{ x_n \right\}$收敛,并求其极限值。
	
	\textbf{3.}设数列$\left\{ x_n \right\}$满足$0<x_1<3$,$x_{n+1}=\sqrt{x_n\left( 3-x_n \right)}$,$n\in N_+$,证明数列$\left\{ x_n \right\}$极限存在并求其极限值。
	
	\textbf{4.}证明数列$\sqrt{2},\sqrt{2+\sqrt{2}},\sqrt{2+\sqrt{2+\sqrt{2}}},\cdots$极限存在,并求其极限值。
	
	\textbf{5.}设$a_1>0$,$\left\{ a_n \right\}$满足$a_{n+1}=\ln \left( 1+a_n \right)$,$n\in N_+$。
	
	(1)证明数列$\left\{ a_n \right\}$极限存在,并求其极限值;
	
	(2)计算
	$$
	\lim_{n\rightarrow \infty} \frac{a_na_{n+1}}{a_n-a_{n+1}}
	$$
	
	\textbf{6.}设数列$\left\{ x_n \right\}$满足$x_{n+1}=\cos x_n$,$n\in N_+$,$x_1=\cos x$,证明该数列极限存在且其极限值为$\cos x-x=0$的根。
	
	\textbf{7.}
	
	(1)证明方程$\tan x=x$在$\left( n\pi ,n\pi +\frac{\pi}{2} \right)$内存在实根$\xi _n$,$n\in N_+$;
	
	(2)计算极限
	$$
	\lim_{n\rightarrow \infty} \left( \xi_{n+1}-\xi _n \right)
	$$
	
	\textbf{8.}
	
	(1)证明方程$e^x+x^{2n+1}=0$在$\left( -1,0 \right)$有唯一实根$x_n$,且$n\in N_+$;
	
	(2)证明数列$\left\{ x_n \right\}$极限存在并求其值$a$;
	
	(3)计算
	$$
	\lim_{n\rightarrow \infty} n\left( x_n-a \right)
	$$
	
	\textbf{9.}设数列$\left\{ x_n \right\}$满足$x_0=a$,$x_1=b,x_{n+1}=\frac{1}{2}\left( x_n+x_{n-1} \right)$,$n\in N_+$,证明数列$\left\{ x_n \right\}$极限存在并求其极限。
	
	\textbf{10.}
	
	(1)证明方程$x^n+x^{n-1}+\cdots +x=1\left( n>1,n\in Z \right)$在区间$\left( \frac{1}{2},1 \right)$内有且仅有一个实根;
	
	(2)记(1)中的实根为$x_n$,证明数列$\left\{ x_n \right\}$极限存在并求其极限。
\end{document}